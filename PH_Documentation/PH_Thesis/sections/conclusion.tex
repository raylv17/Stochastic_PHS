\section{Conclusion}

We were successfully able to show the microscopic force-based pedestrian dynamics model through port-Hamiltonian formulation. We witnessed a shift in perspective from microscopic models, where we inspect agent interactivity on an individual level; to a holistic description of the system-wide interactivity through measuring the system energy. This energy-based perspective, enabled by port-Hamiltonian systems, allowed us to not only observe collective behavior, but also to identify it through quantitative means using the Hamiltonian and $H^*$. With the port-Hamiltonian formulation, we were able to replicate the findings from \cite{tordeux2022multi} of well-known collective phenomenon such as lane and stripe formation. 

We can also observe that the interaction between the pedestrians is isotropic, this is due to the skew-symmetric nature of Hamiltonian systems. A physical interpretation of this, in the context of our model, assumes that the pedestrians interact equally with all of its surroundings, hence there are no vision cones effects \cite{helbing1995social} that may produce bias in interaction in the direction of motion. An attempt to incorporate such anisotropic effects within the port-Hamiltonian approach includes the use of state-dependent input terms \cite{tordeux2022multi} though it remains an active part of research.

Furthermore, we were also able to show the impact of induced noise on these collective phenomenon, in particular, "noise-induced ordering" for stripe formation \cite{khelfa2021initiating,d2021canard}. In the case of lane formation, the collective phenomenon was disrupted, resulting in slower velocities of agents, which is in accordance with the "freezing by heating" phenomenon \cite{helbing2000freezing}, however formation of the clusters of agents could not be reproduced.

With this thesis, we've been able to show that port-Hamiltonian systems, although a relatively young area of research, promises a modeling approach to quantify collective dynamics. Because it models systems from the perspective of energy-based interactions, it is more abstract, and thus useful to understand the underlying dynamics of systems from to various disciplines.