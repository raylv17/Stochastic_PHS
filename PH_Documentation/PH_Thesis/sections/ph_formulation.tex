\section{Port-Hamiltonian Formulation of the Microscopic Model}

This section serves to represent the microscopic dynamics of our pedestrian model using the framework of port-Hamiltonian systems, the generalized finite-dimensional PH formulation is given as:
\begin{equation}
\begin{aligned}
    \dot z(t) &= (J-R)\nabla H(z(t)) + Bu(t) \\
    y(t) &= B^T \nabla H(z(t))
\end{aligned}
\label{eq:ph_generalized}
\end{equation}

\subsection{Preliminaries}
Given a port-Hamiltonian System (PHS), such as the one defined in \autoref{eq:ph_generalized}, it is able to gain energy from or dissipate energy to other interacting systems. Here, $z$ is the state of the system, $u$ is the input, and $y$ is the output of the system. Furthermore, $J \in \mathbb R^{n \times n}$ is a skew-symmetric matrix, $R \in \mathbb R^{n \times n}$ is a positive semi-definite matrix that represents dissipation from the system, and $B \in \mathbb R^{n \times m}$ where $ m \ll n$ being the coefficient for the input.

One can achieve the classical Hamiltonian representation by assuming a system with no input, i.e. $B = 0$, and  no dissipation $R = 0$ as shown below using the first equation from \autoref{eq:ph_generalized}
\begin{gather*}
    \begin{aligned}
    \dot z(t) &= J \cdot \nabla H \\
    \begin{bmatrix}
        \dot q(t) \\ \dot p(t)
    \end{bmatrix}
    &=
    \begin{bmatrix}
        0 & I_n \\ 
        -I_n & 0
    \end{bmatrix}
    \begin{bmatrix}
        \partial H / \partial q \\
        \partial H / \partial p
    \end{bmatrix} 
    \end{aligned}
\end{gather*}
By setting the block matrices $I$ to size $n=1$, it can be demonstrated that the above equations yield the classical Hamiltonian equations, signifying that the skew-symmetry of $J$ is responsible for the conservation of the system.
\begin{align*}
\dot q = \dfrac{\partial H}{\partial p}, \quad \dot p = -\dfrac{\partial H}{\partial q}
\end{align*}
Having $R > 0$ allows the system to dissipate energy and the input $u$ allows the system to gain energy. As the inputs and outputs allows the system to interact with other systems, these can be denoted as ports of the system.

\subsection{Port-Hamiltonian Model}
For a system representing the microscopic model of $N > 2$ pedestrians, the state $z(t)$ would contain the relative positions and velocities of all pedestrians at a given time: $z(t) = (Q,p)^T$, where $Q = (Q_1,\dots,Q_N)$ and $p = (p_1,\dots,p_N)$ are vectors of relative positions and velocities of all pedestrians respectively. Such a system, as shown in \autoref{eq:micro} can be described through port-Hamiltonian formulation presented in \autoref{eq:ph_generalized} as follows:

\begin{equation}
    \begin{aligned}
        \dot z(t) &= (J-R)\nabla H(z(t)) + \lambda \tilde u(t), &\qquad z(0) = (Q^0,p^0)\\
        y(t) &= \lambda \nabla H(z(t)) 
    \end{aligned}
    \label{eq:ph_model}
\end{equation}

Here,
\begin{gather}
    \dot z(t) = 
    \begin{bmatrix}
        Q_1 & \cdots & Q_N & p_1 & \cdots & p_N 
    \end{bmatrix}^T
    \\
    \tilde u = \begin{bmatrix} 0 & \cdots & 0 & u_1(t) & \cdots & u_N(t) \end{bmatrix}^T
    \\
    J =
    \begin{bmatrix}
        0 & M \\
        -M^T & 0 
    \end{bmatrix},
    \quad 
    R =
    \begin{bmatrix}
        0 & 0 \\
        0 & \lambda I
    \end{bmatrix},
    \quad
    M = 
    \begin{bmatrix}
        M_1 \\ \vdots \\ M_N
    \end{bmatrix}
    \label{eq:JRM}
\end{gather}
The resulting system is a linear input-state-output port-Hamiltonian system with dissipation. The Hamiltonian can now be defined as the total energy of the system, i.e. the sum of kinetic and potential energies
\begin{align}
    H(z(t)) = \dfrac{1}{2}||p(t)||^2 + \dfrac{1}{2}\sum_{i=1}^N\sum_{j=1}^NU(Q_{ij}(t))
    \label{eq:Hamiltonian}
\end{align}
The gradient of the Hamiltonian would then be
\begin{gather}
    \nabla H = 
    \begin{bmatrix}
        \partial H / \partial Q \\ \partial H / \partial p
    \end{bmatrix}
    \label{eq:Hamiltonian_partial}
\end{gather}

The matrix equation representation of \autoref{eq:ph_model} can be written as
\begin{gather}
    \begin{bmatrix}
        \dot Q \\ \dot p
    \end{bmatrix} =
    \left(
    \begin{bmatrix}
        0 & M \\ 
        -M^T & 0
    \end{bmatrix} -
    \begin{bmatrix}
        0 & 0 \\
        0 & \lambda I
    \end{bmatrix} \right)
    \begin{bmatrix}
        \frac{1}{2} \nabla U(Q) \\
        p 
    \end{bmatrix}
    + \lambda 
    \begin{bmatrix}
        0 \\ u
    \end{bmatrix}
    \label{eq:matrixeq_ph}
\end{gather}


The block matrix $M$ can be defined such that 
\begin{equation}
\begin{aligned}
    \dot Q = Mp
\end{aligned}
\label{eq:Qmp}   
\end{equation}
with its elements
\begin{gather}
    M_1 = 
    \begin{bmatrix}
        1 & -1 & 0 & 0 & \dots & 0 \\
        1 & 0 & -1 & 0 & \dots & 0 \\
        \vdots & & &  &  & \vdots \\
        1 & & & &  & -1 \\
    \end{bmatrix}, \quad
    M_2 = 
    \begin{bmatrix}
        -1 & 1 & 0 & 0 & \dots & 0 \\
        0 & 1 & -1 & 0 & \dots & 0 \\
        \vdots & \vdots & &  &  & \vdots \\
        0 & 1& & &  & -1 \\
    \end{bmatrix}, \quad \dots
    \label{eq:M_n_def}
\end{gather}
 
%  which reiterates the previously established relationship between pedestrian $i$ and all pedestrians $j \neq i$,
%  \begin{equation*}\begin{aligned}
%     \dot Q_{i} = p_i - p_j
%  \end{aligned}
%  \label{eq:Qdot}
% \end{equation*}
\hrulefill

\textbf{R is positive semi-definite}

 One can show that $R$ is positive semi-definite by showing $x^TRx \geq 0$, where $x \in \mathbb R^n$ is a real-valued vector $ x = [x_1,\dots x_N]$:
 \begin{gather*}
    \begin{aligned}
    x^TRx&=x^T\begin{bmatrix}
        0 & 0 \\
        0 & \lambda I
    \end{bmatrix}x
    \\
    &=\begin{bmatrix}
        x_1\dots x_N
    \end{bmatrix}
    \begin{bmatrix}
        0 & 0 \\
        0 & \lambda I
    \end{bmatrix}
    \begin{bmatrix}
        x_1 \\ \vdots \\ x_N
    \end{bmatrix}
    \\
    &=\begin{bmatrix}
        0 & \lambda x_2 & \dots & \lambda x_N
    \end{bmatrix}
    \begin{bmatrix}
        x_1 \\ \vdots \\ x_N
    \end{bmatrix}
    \\
    &=\lambda(x_2^2+\dots+x_N^2) \geq 0
    \end{aligned}
 \end{gather*}
 Since $\lambda \geq 0$ and the squared terms are also positive, we can conclude that last expression as a whole is also greater than or equal to zero, proving that $x^TRx \geq 0$

 It can be observed that the parameter $\lambda$ is present in both the dissipation term $R$ and the coefficient term of the input $\tilde u$.
 
\hrulefill

\textbf{Example: $N=3$}

To better illustrate the implementation of the model, here we will concretely show the formulation of the model using $N=3$ pedestrians. Here the matrix $M = (M_1, M_2, M_3)^T$, where each $M_i$ shows the difference in quantities with respect to the pedestrian $i$ as shown below.

\begin{gather*}
    M_1 = 
    \begin{bmatrix}
        1 & -1 & 0 \\
        1 & 0 & -1 
    \end{bmatrix},\;
    M_2 = 
    \begin{bmatrix}
        -1 & 1 & 0 \\
        0 & 1 & -1 
    \end{bmatrix},\;
    M_3 = 
    \begin{bmatrix}
        -1 & 0 & 1 \\
        0 & -1 & 1 
    \end{bmatrix}
\end{gather*}

It can be observed that $M$ unfolds the relative velocities of pedestrians, resembling the dynamics presented in the first equation of \autoref{eq:micro}
% \end{column}
% \begin{column}{0.5\textwidth}
\begin{gather*}
    M = 
    \begin{bmatrix}
        M_1 \\ 
        M_2 \\
        M_3
    \end{bmatrix}, \qquad
    \dot Q = Mp = 
    \begin{bmatrix}
        p_1 - p_2 \\
        p_1 - p_3 \\
        p_2 - p_1 \\
        p_2 - p_3 \\
        p_3 - p_1 \\
        p_3 - p_2 
    \end{bmatrix}
\end{gather*}
Lastly, we represent the Hamiltonian
\begin{gather*}
    \begin{aligned}
        H = \frac{1}{2}(p_1^2 + p_2^2 + p_3^2) + \frac{1}{2}(&\nabla U_{12} + \nabla  U_{13} + \\  
        &\nabla  U_{21} +\nabla  U_{23} + \\ 
        &\nabla  U_{31} +\nabla  U_{32})
    \end{aligned}    
\end{gather*}

\hrulefill

\subsection{Deriving the Equations of Motion}
We can show that the port-Hamiltonian formulation \autoref{eq:ph_model} is indeed the same model as \autoref{eq:micro} by deriving the latter from the former. 
Let us start by simplifying the matrix equation \autoref{eq:matrixeq_ph} for $\dot Q$ and $\dot p$

\begin{gather*}
    \dot Q = 
    \left[
    \begin{bmatrix}
        0 & M
    \end{bmatrix}
    - \begin{bmatrix}
        0 & 0
    \end{bmatrix}
    \right]
    \begin{bmatrix}
        \frac{1}{2}\nabla U(Q) \\
        p
    \end{bmatrix}
    +\lambda\begin{bmatrix}
        0
    \end{bmatrix}
\end{gather*}
\begin{gather*}
    \dot p= 
    \left[
    \begin{bmatrix}
        -M^T & 0
    \end{bmatrix}
    - \begin{bmatrix}
        0 & \lambda I
    \end{bmatrix}
    \right]
    \begin{bmatrix}
        \frac{1}{2}\nabla U(Q) \\
        p
    \end{bmatrix}
    +\lambda\begin{bmatrix}
        u
    \end{bmatrix}
\end{gather*}
Resolving $\dot Q$ would reiterate the relationship described in \autoref{eq:Qmp}
\begin{equation*}
    \dot Q = Mp
\end{equation*}
And, resolving $\dot p$ gives us the equation for the velocity in matrix form
\begin{equation*}
    \dot p = -M^T\frac{1}{2}\nabla U(Q) - \lambda p + \lambda u
\end{equation*}

Using the definition of $M$ from \autoref{eq:JRM} and \autoref{eq:M_n_def}, we can construct the equations for $Q$ for every $i = 1,\dots,N$.
\begin{align*}
    \dot Q_{ij} = p_i - p_j
\end{align*}
Similarly, using the property that $U(x)$ is an odd function \autoref{eq:def_U}, we can derive $\dot p$
\begin{align*}
    \dot p_i &= \lambda(u_i - p_i) - \frac{1}{2}\sum_{j \neq i}(\nabla U(Q_{ij}) - \nabla U(Q_{ji})) \\ 
    \dot p_i &= \lambda(u_i - p_i) - \sum_{j \neq i}\nabla U(Q_{ij}) 
\end{align*}

By showing the derivation of the same microscopic dynamics from the port-Hamiltonian formulation, we can establish that the proposed microscopic model for the pedestrian dynamics has Hamiltonian structure, and that the system can be described using the Hamiltonian $H$ as a quantitative measure for the collective dynamics of the pedestrians.

\subsection{Stochastic Port Hamiltonian Formulation}
Continuing from the stochastic induced dynamics of the microscopic model in \autoref{eq:stoch_micro}, we can use the port-Hamiltonian formulation of the deterministic dynamics \autoref{eq:ph_model} and represent it with the noise term.
\begin{equation}
    \begin{aligned}
        dz(t) &= (J-R)\nabla H(z(t))dt + \lambda \tilde u(t)dt + \sigma d\xi(t)\\
        dy(t) &= \lambda \nabla H(z(t))dt
    \end{aligned}
    \label{eq:stoch_ph_model}
\end{equation}
Similar to \autoref{eq:matrixeq_ph}, and following the formulation from \cite{rudiger2024stability}, the matrix equation representation would become
\begin{gather}
    \begin{bmatrix}
        dQ \\ dp
    \end{bmatrix} =
    \left(
    \begin{bmatrix}
        0 & M \\ 
        -M^T & 0
    \end{bmatrix} -
    \begin{bmatrix}
        0 & 0 \\
        0 & \lambda I
    \end{bmatrix} \right)
    \begin{bmatrix}
        \frac{1}{2} \nabla U(Q) \\
        p 
    \end{bmatrix}dt
    + \lambda 
    \begin{bmatrix}
        0 \\ u
    \end{bmatrix}dt
    + \sigma
    \begin{bmatrix}
        0 \\ dW(t)
    \end{bmatrix}
    \label{eq:matrix_stoc_eq_ph}
\end{gather}

This will be used as the basis of our stochastic system. Following from the dynamics presented in \autoref{eq:stoch_micro}, it can be noted that the stochastic term is only effecting the momentum equation.

It can also be noted that the expression for the Hamiltonian \autoref{eq:Hamiltonian} will remain the same, since the stochastic terms are included within the momentum term $\dot p$. 

\subsection{Macroscopic Observations from Hamiltonian Behavior}
The time derivative of the Hamiltonian is derived by its taking the total derivative with respect to time:
\begin{align*}
    \dfrac{d}{dt}H(z(t)) = \nabla^T H(z(t))\cdot \dot z(t)
\end{align*}
Substituting $\dot z(t)$ from \autoref{eq:ph_model} denoting that $y = \lambda \nabla H$ and thus $y^T = \nabla^T H \lambda$, we get
\begin{align*}
    \dfrac{d}{dt}H(z(t)) = \nabla^T H(z(t))(J-R)\nabla H(z(t)) + \underbrace{\nabla^T H(z(t)) \lambda}_{y^T} \tilde u
\end{align*}
Using the skew-symmetric property of $J$ that $x^TJx=0$ we can simplify and obtain the expression for the time derivative of the Hamiltonian.
\begin{align}
    \frac{d}{dt}H(z(t)) = y^Tz(t)\tilde u(t) - \nabla^T H(z(t))R \nabla H(z(t))
\end{align}

Expanding the terms further by substituting the gradient of the Hamiltonian \autoref{eq:Hamiltonian_partial} and the definition of $R$ \autoref{eq:JRM} can help us further simplify the expression giving us the following energy balance
\begin{align}
    \dfrac{d}{dt}H(z(t)) &= \lambda \nabla^T H(z(t)) \tilde u - \nabla^T H(z(t)) R \nabla H(z(t)) \nonumber \\
    &\begin{aligned}
        &= \lambda 
        \begin{bmatrix}
        % \dfrac{\partial H}{\partial Q} & \dfrac{\partial H}{\partial p}
        \partial H / \partial Q & \partial H / \partial p
        \end{bmatrix}
        \begin{bmatrix}
            0 \\
            u
        \end{bmatrix}
        - 
        \begin{bmatrix}
            % \dfrac{\partial H}{\partial Q} & \dfrac{\partial H}{\partial p}
        \partial H / \partial Q & \partial H / \partial p
        \end{bmatrix}
        \begin{bmatrix}
            0 & 0 \\ 0 & \lambda I
        \end{bmatrix}
        \begin{bmatrix}
            % \frac{\partial }{\partial Q} H \\ \frac{\partial }{\partial p}H
        \partial H / \partial Q \\ \partial H / \partial p
        \end{bmatrix}  \\ 
        &= \lambda p^T u - \lambda  p^T p \\ 
        &= \lambda p^T (u - p) \\ 
    \end{aligned} \nonumber \\
    \dfrac{d}{dt}H(z(t))&= \lambda \langle p(t),u(t)-p(t) \rangle
    \label{eq:dH}
\end{align}


It can be seen that the time derivative of the Hamiltonian $\frac{d}{dt}H(z(t))$ depends only on the velocities of the pedestrians.

With this, we can confirm the following claims
\begin{itemize}
    \item The Hamiltonian remains constant $\frac{d}{dt}H(z(t)) = 0$ for all times $t \geq 0$ if the system has no dissipation $\lambda = 0$. 
    \begin{align*}
        \forall t \geq 0,\quad \dfrac{d}{dt}H(z(t)) = 0 \qquad \text{if} \qquad \lambda = 0
    \end{align*}
    Reiterating that purely Hamiltonian behavior is indeed conservative.
    \item If the desired velocities are zero, i.e. $u = 0$, then the Hamiltonian decreases over time
    \begin{align*}
        \forall t \geq 0,\quad \dfrac{d}{dt}H(z(t)) \leq 0 \qquad \text{if} \qquad \forall i,\; u_i = 0
    \end{align*}
    Since the system is allowed to dissipate without input feed, the asymptotic behavior would yield crystallization, i.e. the pedestrians would stop moving.
    \item If all the pedestrians move at their desired velocities, i.e. $p(t) = u(t)$, then the time derivative is zero
    \begin{align*}
        \dfrac{d}{dt} H(z(t) = 0 \qquad \text{if} \qquad \forall i,\; p_i(t) = u_i(t)
    \end{align*}
\end{itemize}