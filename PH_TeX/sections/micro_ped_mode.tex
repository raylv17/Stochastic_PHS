\section{Microscopic Pedestrian Model}

We will first construct the pedestrian model focusing on the interaction of agents to each other i.e. in the microscopic scale. In contrast, the macroscopic scale would focus on the resulting dynamics of the agents' interactions as a collective, which will be presented in more detail in the next section.

\subsection{Pedestrian Attributes}
Let us first start by describing our model. We follow the approach presented in \cite{tordeux2022multi} which follows a general class of microscopic force-based models approach given in \cite{helbing1995social,chraibi2011force}. The pedestrians exist on a toroidal surface, in other words, our model has periodic boundaries. For simplicity, the mass for all pedestrians is set to 1, hence their momentum $p_i$ is equal to the velocity, i.e. $p_i = m_i v_i = v_i$. The pedestrians also posses a desired velocity, which is assigned to each of them as an input to the model, this directs the pedestrians to desire a certain direction to move towards during the simulation run.

Given a pedestrian $i$ in $\mathbb{R}^2$ and time $0\leq t \leq T$, it possesses the following attributes:
\begin{itemize}
    \item Desired velocity: $u_i(t): [0,T] \mapsto \mathbb{R}^2$ 
    \item Current velocity: $p_i(t): [0,T] \mapsto \mathbb{R}^2$
    \item Current position: $q_i(t): [0,T] \mapsto \mathbb{R}^2$
\end{itemize} 
\begin{listing}[!ht]
\begin{minted}[escapeinside=??, frame=single]{julia}
## Define Agent
using Agents
@agent struct Pedestrian(ContinuousAgent{2,Float64})
    u?\sub{i}?::Vector{Float64} # Desired Velocity
end
\end{minted}
\caption{Defining the pedestrian agent in Julia's Agents.jl package. It is to be noted that \texttt{ContinuousAgent} specifies that our Pedestrian is a continuous agent with predefined position and velocity attributes constructed within the \texttt{@agent} macro. We only need to declare additional attributes such as \texttt{u}\sub{i}} 
\end{listing}

For a model with $N \geq 2$ pedestrians, the relative position $\dot Q_{ij}(t)$ of a pedestrian $i$ to pedestrian $j$ is denoted as follows
\begin{align*}
    {Q}_{ij}(t) = q_i - q_j
\end{align*}
The vector of relative positions $Q_i(t)$ of a pedestrian $i$ to all pedestrians $j \neq i$ is denoted as follows
\begin{align*}
    {Q}_{i}(t) = ({Q}_{ij}(t))_{i\neq j} : [0,T] \mapsto \mathbb{R}^{2\cdot(N-1)} \text{ for } j = 1,\dots, N, \quad j \neq i
\end{align*}


\subsection{Microscopic Model Dynamics}
\label{section:micro_model_dynamics}
The dynamics of the pedestrians are based on (i) short-range repulsion $U$ among pedestrians based on their distances from others and (ii) attraction $\lambda$ toward a desired velocity $u_i$. The microscopic dynamics for the pedestrian $i$ is given by the following equations:
\begin{equation}
\begin{aligned}
    \dot Q_{i}(t) &= p_i(t) - p_j(t), &j = 1,\dots,N \quad j \neq i& & \dot Q_i(0) &= Q^0_i \\
    \dot p_i(t) &= \lambda(u_i(t) - p_i(t)) - \sum_{j \neq i} \nabla U (Q_{ij}(t)), && & \dot p_i(0) &= p^0_i
\end{aligned}
\label{eq:micro}
\end{equation}
Here,
\begin{itemize}
    \item $\lambda \in \mathbb{R}_{\geq 0}$, is a parameter for the relaxation rate. In other words, it determines the intensity to reach the desired velocity. 
    \item $U(Q_{ij})$, is a nonlinear repulsive interaction potential. This determines the magnitude of repulsion of a pedestrian $i$ to other pedestrians $j$ as a function of relative position $Q_i = q_i - q_j$. It is defined as follows 
    \begin{align} 
        U(x) : \mathbb{R}^2 \mapsto \mathbb{R}, \quad\quad U(x) = ABe^{-|x|/B}
        \label{eq:def_potU}
    \end{align}
    Here, $|\cdot|$ is the minimum euclidean distance on the torus. $A$ and $B$ are scalar-valued parameters for the repulsion strength and the range of interaction respectively.
\end{itemize}

Hence, $\nabla U(x)$ is defined as
\begin{align} 
    \nabla U(x) = -\dfrac{x}{|x|}Ae^{-|x|/B} = -\nabla U(-x)
    \label{eq:def_U}
\end{align}
It can be seen that this parameter is responsible for avoiding collisions between pedestrians, ensuring that the repulsive forces increase in magnitude the closer a pedestrian is. The underlying assumption is that the repulsive forces are a function of distance only. This is adapted from the concept of social forces and how they are determined from the surroundings of pedestrians.

\subsection{Stochastic Model Dynamics}

To study the stochastic variant of our model, and how they effect the overall dynamics, we will introduce stochastic terms into our model. Generally a typical stochastic differential equation (SDE) is of the form
\begin{equation*}
    dX_t = \underbrace{f(X_t, t)dt}_{\text{drift}} + \underbrace{g(X_t, t)dW_t}_{\text{diffusion}}
\end{equation*}
Which is a combination of a deterministic component, denoted as the \textit{drift} term; and a stochastic component, denoted as the \textit{diffusion} term. In our case, we introduce an additive noise term $\sigma dW_i(t)$ to our deterministic model from before \autoref{eq:micro} resulting in \autoref{eq:stoch_micro}. To be able to have a better physical interpretation of the model, the noise is introduced in the momentum term, rather than the position term. So one can interpret this as agents moving randomly while maintaining a continuous motion, as opposed to randomly teleporting in different locations at every time-step.
\begin{gather}
\begin{aligned}
    dQ_{i}(t) &= (p_i(t) - p_j(t))dt \\
    dp_i(t) &= \lambda(u_i(t) - p_i(t))dt - \sum_{j \neq i} \nabla U (Q_{ij}(t))dt + \sigma dW_i(t)
    \label{eq:stoch_micro}
\end{aligned}
\end{gather}

The equation is reminiscent of a typical SDE with the first two terms being the deterministic part of the equation, i.e. drift and the last term being the stochastic part, i.e. the diffusion term. The randomness is induced via the Wiener process $(W_i(t))^N_{i=1}: [0,\infty) \times \Omega \rightarrow \mathbb{R}^N$, which is a mathematical description of the Brownian motion, that generates a sequence of random variables defined on a probability space $(\Omega, F, P)$ with independent increments \cite{oksendal2003stochastic}. Here, $\Omega$ is the sample space, $F$ the event space, and $P$ the probability measure. The stochastic term is amplified by the constant $\sigma \in \mathbb{R}$, which is often denoted as the volatility or diffusion coefficient.


With the description of the models at hand, we can easily simulate them. However, we currently only have information about the individual agents. What we need is a description of the model dynamics as a whole, a measure with which we can describe the macroscopic behaviors, and identify collective phenomenon such as the previously mentioned lane and stripe formation. This measure, as we will see in the next section, is the Hamiltonian, i.e. the total energy of the system for our case.