\section{Port-Hamiltonian Formulation}

\begin{frame}{Port-Hamiltonian Systems}
    \begin{align*}
        \dot{z} &= (J-R)\nabla H(z(t)) + Gu(t) \\
        y &= G^T\nabla H(z(t))
    \end{align*}
    Here, \cite{van2006port}
    \begin{itemize}
        \item $J\in \mathbb{R}^{n \times n}$, a skew-symmetric matrix
        \item $R\in \mathbb{R}^{n \times n}$, a positive semi-definite matrix
        \begin{itemize}
            \item $R \equiv 0$ and $G \equiv 0$ results in a conservative Hamiltonian System
            \item $R > 0$ results in a dissipative system 
        \end{itemize}
        \item $H$, Hamiltonian, total energy of the system
        \begin{itemize}
            \item $\nabla H \in \mathbb{R}^{n} = \begin{bmatrix} \partial_q H & \partial_p H \end{bmatrix}^T$
        \end{itemize}
        \item $G\in \mathbb{R}^{n \times m}$, coefficient for input $u$
        \item u, input parameter
        \item y, output parameter (to other interacting systems)
    \end{itemize}
\end{frame}


\begin{frame}{Port-Hamiltonian Formulation of the microscopic force-based pedestrian model}
    \begin{itemize}
        \item $\displaystyle{ H(z(t)) = \dfrac{1}{2}||p||^2 + \dfrac{1}{2}\sum^N_{i=1}\sum^N_{j\neq i}U(Q_{ij}(t))}$
        \item $G = \lambda \in \mathbb{R}$
    \end{itemize}
    
    Port-Hamiltonian Formulation for the pedestrian model \cite{tordeux2022multi}
    \begin{gather*}
        \begin{bmatrix}
            \dot Q \\
            \dot p
        \end{bmatrix}
        = 
        \underbrace{
        \begin{pmatrix}
            \begin{bmatrix}
                0     & M \\
                -M^T  & 0
            \end{bmatrix}
            -
            \begin{bmatrix}
                0     & 0 \\
                0     & \lambda I
            \end{bmatrix}
        \end{pmatrix}
        }_{(J-R)}
        \underbrace{
        \begin{bmatrix}
            \frac{1}{2}\nabla U(Q) \\
            p
        \end{bmatrix}
        }_{\nabla H}
        +
        \lambda
        \begin{bmatrix}
            0 \\
            u
        \end{bmatrix}
    \end{gather*}
    \begin{table}
        \centering
        \begin{tabular}{ll}
            $\dot Q = Mp$ & $\rightarrow\dot Q_{ij} = p_i - p_j$ \\
            
            $\dot p = -M^T\frac{1}{2}\nabla U(Q) - \lambda p - \lambda u$ & $\rightarrow\dot p_i = \lambda(u_i - p_i) - \sum_{j \neq i} \nabla U(Q_{ij})$
        \end{tabular}
    \end{table}
    % \begin{align*}
    %     \dot Q &= p_i - p_j \because \dot Q = Mp\\
        
    % \end{align*}

\end{frame}

\begin{frame}{Example: N=3}
    % \begin{columns}
        % \begin{column}{0.2\textwidth}
    \begin{gather*}
        M_1 = 
        \begin{bmatrix}
            1 & -1 & 0 \\
            1 & 0 & -1 
        \end{bmatrix}
        M_2 = 
        \begin{bmatrix}
            -1 & 1 & 0 \\
            0 & 1 & -1 
        \end{bmatrix}
        M_3 = 
        \begin{bmatrix}
            -1 & 0 & 1 \\
            0 & -1 & 1 
        \end{bmatrix}
    \end{gather*}
    % \end{column}
    % \begin{column}{0.5\textwidth}
    \begin{gather*}
        M = 
        \begin{bmatrix}
            M_1 \\ 
            M_2 \\
            M_3
        \end{bmatrix},
        \dot Q = Mp = 
        \begin{bmatrix}
            p_1 - p_2 \\
            p_1 - p_3 \\
            p_2 - p_1 \\
            p_2 - p_3 \\
            p_3 - p_1 \\
            p_3 - p_2 
        \end{bmatrix}
    \end{gather*}

    \begin{align*}
        H = \dfrac{1}{2}(p_1^2 + p_2^2 + p_3^2) + \dfrac{1}{2}(&\nabla U_{12} + \nabla  U_{13} + \\  
        &\nabla  U_{21} +\nabla  U_{23} + \\ 
        &\nabla  U_{31} +\nabla  U_{32})
    \end{align*}
% \end{column}
    % \end{columns}
\end{frame}
