\section{Overview}

% introduction and overview
% \section{Motivation}
\begin{frame}
    \frametitle{Overview}
    \begin{itemize}
        \item Study of collective behavior
        \item from microscopic force-based pedestrian models
        \item using Port-Hamiltonian Framework
        \item with stochastic elements
    \end{itemize}
\end{frame}
    
% \begin{frame}{Collective Behavior}{Macroscopic Behaviour}
%     In the context of Pedestiran Dynamics
%     \begin{itemize}
%         \item<1-> Counter Flow
%         \item<2-> Cross Flow
%     \end{itemize}
    
% \end{frame}


% \begin{frame}
%     \centering
%     \animategraphics[loop,autoplay,width=4cm]{20}{to_png/counterflow/frame-}{0}{273}
% \end{frame}

\begin{frame}{Collective Behavior}{Macroscopic Behaviour}
    \begin{columns}[t]
\begin{column}{0.48\linewidth}
    \centering
    Counter Flow: \linebreak Lane Formation \linebreak \linebreak
    \animategraphics[loop,autoplay,width=5cm]{25}{to_png/counterflow/frame-}{10}{273}
\end{column}
\begin{column}{0.48\linewidth}
    \centering
    Cross Flow: \linebreak Stripe Formation \linebreak \linebreak
    \animategraphics[loop,autoplay,width=5cm]{25}{to_png/crossflow/frame-}{10}{273}
\end{column}
    \end{columns}    
\end{frame}
%asuming you images are called "something-0.png" up to "something-16.png" 
% \begin{frame}
%     \transduration<0-100>{0}
%     \multiinclude[<+->][format=png, graphics={width=4cm}]{to_png/counterflow/frame}
% \end{frame}